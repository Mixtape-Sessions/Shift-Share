\documentclass{beamer}

\input{preamble.tex}



\begin{document}

\imageframe{./lecture_includes/cover.png}

\section{Motivation}

\begin{frame}{Motivation} 
Many treatments/instruments are SSIV-like: combining multiple sets of variation, w/ \bgSun{some} as-good-as-randomly assigned, but not \bgElectricViolet{all}:

\pause
\begin{enumerate}
  \item Spatial/network/GE spillover treatments: e.g. the number of neighbors selected for a randomized intervention:\pause

  \bgSun{Who got selected for the intervention} \& \bgElectricViolet{who neighbors whom} 

  \smallskip\pause
  \item Regional growth of market access from transportation upgrades:\pause

  \bgElectricViolet{Location} + \bgSun{timing of upgrades} \& \bgElectricViolet{location and size of markets}

  \smallskip\pause
  \item An individual's eligibility for a public program, e.g. Medicaid:\pause

  \bgSun{State-level policy} \& \bgElectricViolet{individual income and demographics}
\end{enumerate}

\bigskip\pause
How can we just leverage the \bgSun{exogenous shocks} to such $z_i$?
\end{frame}

\begin{frame}{Borusyak and Hull (BH, 2022): Main Points}
  
  {\small 
  \vspace{-3mm}
	\begin{enumerate}

	\item \bgElectricViolet{Non-random exposure} to \bgSun{exogenous shocks} generates systematic variation which can lead to omitted variable bias. \pause 
  
  \vspace{-2mm}
  \begin{itemize}
    \item Randomizing roads $\not\Rightarrow$ random market access growth from them\pause 
  \end{itemize}

	\item The systematic variation can be removed via novel ``recentering''\pause 

    \vspace{-2mm}
		\begin{itemize}
		\item Specify many counterfactual sets of \bgSun{shocks} 
    
    \pause 
		\item Compute $\mu_i$, the average $z_i$ across counterfactuals, by simulation \\ \pause\emph{--- the key confounder (similar to a propensity score)}
    
    \pause 
		\item ``Recenter'' $z_i$ by $\mu_i$ (i.e. instrument with $\tilde{z}_i=z_i-\mu_i$) or control for $\mu_i$ 
    
    \pause 
		\item Conventional solutions (e.g. directly instrumenting with shocks or controlling for all features of exposure) are often infeasible 		
		\end{itemize}

  \pause 
	\item Recentering solution also can have attractive efficiency properties
    \vspace{-2mm}
    \begin{itemize}
	  	\item Leverages \bgElectricViolet{non-random exposure} to best predict shock effects
		\end{itemize}
	\end{enumerate}
  }
\end{frame}


\begin{frame}{(Some) Other Settings where these Points are Relevant}
  
{\small
Linear shift-share IV {\small\color{gray}(Autor et al. 2013, Borusyak et al. 2022)}
	
\vspace{0.2cm}
Nonlinear shift-share IV {\small\color{gray}(Boustan et al. 2013, Berman et al. \phantom{Z}2015, Chodorow-Reich and Wieland 2020, Derenoncourt 2021)} 
	
\vspace{0.2cm}
IV based on centralized school assignment mechanisms {\small\color{gray}(Abdulkadiro\u{g}lu et al. 2017, 2019, Angrist et al. 2020)}

\vspace{0.2cm}
Model-implied optimal IV {\small\color{gray}(Ad\~{a}o-Arkolakis-Esposito 2021)}

\vspace{0.2cm}
Weather instruments {\small\color{gray}(Gomez et al. 2007, Madestam et al. 2013)}%

\vspace{0.2cm}
``Free space'' instruments for mass media access {\small\color{gray}(Olken 2009, Yanagizawa-Drott 2014)}

}
\end{frame}

\section{Intuition}

\subsection{Market Access Effects}
\begin{frame}{Example 1: Market Access Effects via RCT} 
\vspace{-0.3cm}
Theory suggests transportation upgrades affect local outcomes (e.g. land value) of regions $i$ by increasing their market access (MA): 

\vspace{-0.7cm}
\begin{gather*}
\Delta\log V_i  = \beta\Delta\log MA_{i} +\varepsilon_{i},\\ 
\text{where } MA_{it} = \sum_j \tau({\color<3->{sun} g_t},{\color<5->{electric-violet}loc_i},{\color<5->{electric-violet}loc_j})^{-1}{\color<5->{electric-violet}pop_{j}}, 
\end{gather*}

\vspace{-0.3cm}%
for {\color<3->{sun} road network $g_t$} in periods $t=1,2$, {\color<5->{electric-violet} region locations $loc_j$} (co-determining travel cost $\tau$), and regional {\color<5->{electric-violet}population $pop_j$}

\vspace{0.2cm}\pause 

Imagine an experiment randomly connecting adjacent regions by road\pause \pause

\begin{itemize}
	\item MA only grows because of the random transportation shocks

	\item So can we view variation in MA growth as random and just run OLS? % like in an RCT
\end{itemize}

\pause
No. Randomizing roads $\not\Rightarrow$ randomizing $MA$ due to them!
\end{frame}

\begin{frame}[t]{Illustration: Market Access on a Square Island} 
\vspace{-0.3cm}
	\begin{center}
		Start from no roads, assume equal population everywhere

		\includegraphics[height=0.8\textheight]{lecture_includes/empty_gridnoroads.pdf}
	\end{center}
\end{frame}

\begin{frame}[t]{Illustration: Market Access on a Square Island} 
\vspace{-0.3cm}
	\begin{center}
		Randomly connect adjacent regions by road \uncover<2->{and compute MA growth}

		\only<1>{\includegraphics[height=0.8\textheight]{lecture_includes/line_only_noroads.pdf}}\only<2>{\includegraphics[height=0.8\textheight]{lecture_includes/dMA_and_line_noroads.pdf}}\only<3>{\includegraphics[height=0.8\textheight]{lecture_includes/dMA_and_line_seed2_noroads.pdf}}\only<4>{\includegraphics[height=0.8\textheight]{lecture_includes/dMA_and_line_seed3_noroads.pdf}}
	\end{center}
\end{frame}

\begin{frame}[t,label=ExpInst_noroads]{Expected Market Access Growth $\mu_i$}
\vspace{-0.3cm}
	\begin{center}
		Some regions get systematically more MA 

		\includegraphics[height=0.8\textheight]{lecture_includes/Exp_dMA_noroads.pdf}
	\end{center}
\end{frame}

\begin{frame}[t]{Illustration: High-Speed Rail in China} 
\vspace{-0.3cm}
	\begin{center}
		149 lines were built or planned (as of April 2019)

		\includegraphics[trim={1cm 0.5cm 0.5cm 1cm},clip,width=11cm]{lecture_includes/Lines_actual_planned_nocolor.png}
	\end{center}
\end{frame}

\begin{frame}[t]{Illustration: High-Speed Rail in China} 
\vspace{-0.3cm}
	\begin{center}
		The 83 lines actually built by 2016. Suppose timing is random

		\includegraphics[trim={1cm 0.5cm 0.5cm 1cm},clip,width=11cm]{lecture_includes/Line_panel2016.png}
	\end{center}
\end{frame}

\begin{frame}[t]{Illustration: High-Speed Rail in China} 
\vspace{-0.3cm}
	\begin{center}
		A counterfactual draw of 83 lines by 2016

		\includegraphics[trim={1cm 0.5cm 0.5cm 1cm},clip,width=11cm]{lecture_includes/Sim_Line_nlink2016.png}
	\end{center}
\end{frame}

\begin{frame}[t]{Illustration: High-Speed Rail in China} 
\vspace{-0.3cm}
	\begin{center}
		Expected MA growth, $\mu_i$

		\includegraphics[trim={1cm 0.5cm 0.5cm 1cm},clip,width=11cm]{lecture_includes/DateExpected2016.png}
	\end{center}
\end{frame}

\begin{frame}[label=RCSolution]{OVB and Recentering Solution}
\vspace{0.2cm}
	Systematic variation in MA growth can generate OVB 
		\vspace{0.1cm}
		\begin{itemize}  
		\item E.g. land values fall in the periphery because of rising sea levels
		\vspace{0.1cm}
		\item More vs less developed Chinese regions may be on different trends \pause 
		\end{itemize}
	\vspace{0.5cm}
	Systematic variation can be removed via ``recentering'':\vspace{-0.25cm}
	$$\hspace{-2.85cm}\begin{matrix}\text{Recentered}\\\text{MA growth}\end{matrix}\ =\ \begin{matrix}\text{Realized}\\\text{MA growth}\end{matrix}\ -\ \begin{matrix}\text{Expected}\\\text{MA growth}\end{matrix}$$\pause
\vspace{-0.2cm}

Recentered MA is a valid instrument for realized MA growth 
		\vspace{0.1cm}
		\begin{itemize}  
		\item Compares MA from actual and counterfactual shocks 
		\item As it turns out, we can also control for expected MA growth
		\end{itemize}
\end{frame}

\begin{frame}{Linear SSIV Redux}

Classic SSIV is a special case where $z_i=\sum_n \electricViolet{s_{in}}\sun{g_n}$ is linear in the exogenous shocks\pause{}
\medskip

The expected instrument is $\mu_i=E\left[\sum_n\electricViolet{s_{in}}\sun{g_n}\mid \electricViolet{s}\right]=\sum_n\electricViolet{s_{in}} E\left[\sun{g_n}\mid \electricViolet{s}\right]$\smallskip\pause{}

\begin{itemize}
\item If $E\left[\sun{g_n}\mid \electricViolet{s}\right]=\gamma$, we need to adjust for $\gamma\left(\sum_n\electricViolet{s_{in}}\right)$\pause{}
\item Linear in the sum-of-shares $S_i=\sum_n\electricViolet{s_{in}}$; it turns out controlling for this observable is enough (recall FWL theorem!)\pause{}
\item If $\sun{g_n}$ is only exogenous conditional on $\electricViolet{q_n}$, with $E\left[\sun{g_n}\mid \electricViolet{s,q}\right]=\electricViolet{q_n}^\prime\gamma$, we need to adjust for $\sum_n\electricViolet{s_{in}} E\left[\sun{g_n}\mid \electricViolet{s,q}\right]=\gamma\left(\sum_n\electricViolet{s_{in}q_n}\right)$\pause{}
\item Controlling for $\sum_n\electricViolet{s_{in}q_n}$ is enough (sound familiar?) 
\end{itemize}

\end{frame}


\subsection{Medicaid Eligibility Effects}
\begin{frame}{Example 2: Effects of Program Eligibility}
Consider the effects of individual eligilibity $x_i$ for Medicaid:
$$y_i=\beta x_i+\varepsilon_i$$ 
where $x_i$ is determined by $i$'s state policy ${\color<2->{sun}g}{}_{\color<3->{electric-violet}\text{state}_i}$ and {\color<3->{electric-violet}demographics}\pause
\vspace{0.05cm}
	\begin{itemize}
	\item Suppose \bgSun{state policies} are as-good-as-random 
	\vspace{0.1cm}\pause
	\item But \bgElectricViolet{pre-determined demographics} are endogeous $\Rightarrow$ OLS biased
	\end{itemize}
\vspace{0.3cm}\pause
Standard ``simulated instruments'' solution (Currie and Gruber (1996)): \\ use state-level variation (avgerage policy generosity across a ``simulated'' group of individuals) as a single IV for $x_i$ 

\begin{itemize}
\item This works, but is likely inefficient: the policy shocks likely have heterogeneous effects across individuals w/different demos
\end{itemize}
\end{frame}

\begin{frame}{Gaining Power from Recentering}
Consider the effects of individual eligilibity $x_i$ for Medicaid:
$$y_i=\beta x_i+\varepsilon_i$$ 
where $x_i$ is determined by $i$'s state policy ${\color{sun}g}{}_{\color{electric-violet}\text{state}_i}$ and {\color{electric-violet}demographics}\pause

The BH approach:
\vspace{0.05cm}
	\begin{itemize}
	\item Formalize the policy experiment as ``all permutations of $\color{sun}g$ across states are equally likely''
  
  \pause
	\item Compute $\mu_i$ = the share of states in which $i$ would be eligible
  
  \pause
	\item Leverage all variation in $x_i$ but recenter by $\mu_i$ (or control for $\mu_i$)
  
  \pause
	\item Yields efficiency gain by better first-stage prediction, e.g. by removing $i$ who are always or never eligible 
	\end{itemize}
\end{frame}

\section{Formal Framework}

\begin{frame}{General Setup}

\vspace{-0.2cm}
We have a model of $y_i=\beta x_i+\varepsilon_i $ for a fixed population $i=1\dots N$
\begin{itemize}
\item In the paper: extensions to heterogeneous effects, other controls, multiple treatments, nonlinear outcome models, panel data...
\end{itemize}

\pause
We have a candidate instrument $z_i = f_i ({\color{sun}g},{\color{electric-violet}w})$, where {\color{sun} $g$} is a vector of shocks; {\color{electric-violet} $w$} collects predetermined variables; $f_i(\cdot)$ are known mappings 
	\begin{itemize}
	\item Applies to any $z_i$ which can be constructed from observed data
	\item Nests reduced-form regressions: $x_i=z_i$
	\item Allows ${\color{sun} g}=({\color{sun} g_1},\dots,{\color{sun} g_K})$ to vary at a different level than $i$
	\end{itemize}
\end{frame}

\begin{frame}{General Setup}
\vspace{-0.2cm}
We have a model of $y_i=\beta x_i+\varepsilon_i $ for a fixed population $i=1\dots N$
\begin{itemize}
\item In BH: extensions to heterogeneous effects, other controls, multiple treatments, nonlinear outcome models, panel data...
\end{itemize}

We have a candidate instrument $z_i = f_i ({\color{sun}g},{\color{electric-violet}w})$, where {\color{sun} $g$} is a vector of shocks; {\color{electric-violet} $w$} collects predetermined variables; $f_i(\cdot)$ are known mappings 

\medskip
\textbf{Assumptions}:
	\begin{enumerate}
	\item Shocks are exogenous: ${\color{sun}g} \perp \varepsilon \mid {\color{electric-violet} w}$
	\item Conditional distribution $G({\color{sun}g}\mid{\color{electric-violet} w})$ is known (e.g. via randomization protocol or uniform across permutations of $\color{sun}g$)
	\end{enumerate}

\end{frame}

\begin{frame}{Main Results}
	The expected instrument, $\mu_i=E[f_i({\color{sun}g},{\color{electric-violet}w})\mid {\color{electric-violet}w}]\equiv\int f_i({\color{sun}g},{\color{electric-violet}w})dG({\color{sun}g}\mid {\color{electric-violet}w})$, is the sole confounder generating OVB:
%	\vspace{-0.1cm}
	$$E\left[\frac{1}{N}\sum_i z_i \varepsilon_i\right]=E\left[\frac{1}{N}\sum_i\mu_i \varepsilon_i\right]\ne 0\text{, in general}$$
	\pause\vspace{-0.2cm}
	The \emph{recentered instrument} $\tilde{z}_i=z_i-\mu_i$ is a valid instrument for $x_i$:
%	\vspace{-0.1cm}
	$$E\left[\frac{1}{N}\sum_i \tilde z_i \varepsilon_i\right]=0$$ % provided it has a first stage
%	It thus identifies $\beta$, given a first stage ($\expec{\frac{1}{L}\sum_i \tilde z_i x_i}\neq 0$)
	\pause
	Regressions which control for $\mu_i$ also identify $\beta$ (implicitly recenter, by the FWL theorem)
\end{frame}

\begin{frame}{Extensions}
	\textbf{Consistency}: follows when $\tilde z_i$ is weakly mutually dependent across $i$
	\vspace{0.25cm}

	 \textbf{Robustness} to heterogeneous treatment effects: $\tilde{z}_i$ identifies a convex avg. of $\beta_i$ under appropriate first-stage monotonicity
	\vspace{0.25cm}

	 \textbf{Randomization inference} provides exact confidence intervals for $\beta$ (under constant effects) and falsification tests 
	\vspace{0.25cm}

	BH also characterize the \textbf{asy.~efficient} recentered IV among all $f_i(\cdot)$ 

\end{frame}

\section{Applications}

\subsection{Market Access Effects}
\begin{frame}[label=HSRApplication]{App. 1: Market Access from Chinese High-Speed Rail}
	BH first show how instrument recentering can address OVB when estimating the effects of market access growth
	
	\vspace{0.4cm}
	\textbf{Setting}: Chinese HSR; 83 lines built 2008--2016, 66 yet unbuilt
	\vspace{0.1cm}
	\begin{itemize}
	\item Market access: $MA_{i t}= \sum_k \exp\left(-0.02\tau_{i k t}\right) p_{k,2000}$, where $\tau_{i k t}$ is HSR-affected travel time between prefecture capitals (Zheng and Kahn, 2013) and $p_{i,2000}$ is prefecture $i$'s population in 2000
	\vspace{0.1cm}
	\item Relate to employment growth in 274 prefectures, 2007-2016
	\end{itemize}
\end{frame}
	

\begin{frame}{Simple OLS Regressions Suggest a Large MA Effect} 
	\begin{center}
	\includegraphics[width=\textwidth]{lecture_includes/emp_ols_binscatter.pdf}
	\end{center}
\end{frame}

\begin{frame}[t,label=LinePanel]{High vs. Low MA Growth is Not a Convincing Contrast!}
	\begin{center}
	\includegraphics[trim={1cm 0.5cm 0.5cm 1cm},clip,width=11cm]{lecture_includes/Line_panel2016.png}
	\end{center}
\end{frame}

\begin{frame}[label=HSR]{How to Find Valid Treatment-Control Contrasts?}
Add controls (province FE, longitude, etc...)
\vspace{0.05cm}
	\begin{itemize}
	\item Hard to justify \emph{ex ante} since MA is a variable constructed based on a structural model
	\vspace{0.1cm}
	\item No experimental analog
	\end{itemize}

\vspace{0.4cm}\pause
Find valid contrasts for \emph{one} source of variation---a natural experiment
\vspace{0.05cm}
	\begin{itemize}
	\item Bartelme (2018): shocks affecting market size
	\vspace{0.1cm}
	\item Donaldson (2018): built vs unbuilt lines
	\vspace{0.1cm}
	\item BH application: assume random timing of observably similar lines 
	\end{itemize}
\end{frame}

\begin{frame}[t,label=BuiltPlanned]{Built and Planned HSR Lines}
\vspace{-0.5cm}
	\begin{center}
	BH reshuffle built \& planned lines connecting the same \# of regions 
	
	\includegraphics[trim={1cm 0.5cm 0.5cm 1cm},clip,width=11cm]{lecture_includes/Lines_actual_planned.png}
	\end{center}
\end{frame}

\begin{frame}[t,label=HSRExpected]{Expected Market Access Growth (2007--2016), $\mu_i$}
	\begin{center}
	
	\includegraphics[trim={1cm 0.5cm 0cm 1cm},clip,width=11cm]{lecture_includes/NlinkExpected2016.png}
	
	\phantom{X}
	\end{center}
\end{frame}


\begin{frame}{Recentered Market Access Growth (2007--2016), $\tilde z_i$}
	\begin{center}
	\includegraphics[trim={1cm 0.5cm 0cm 1cm},clip,width=11cm]{lecture_includes/NlinkRecentered2016.png}
	\end{center}
\end{frame}

\begin{frame}[label=HSRSpecTests]{Market Access Balance Regressions}
\begin{center}
\vspace{-0.4cm}
\includegraphics[width=12cm]{lecture_includes/spec_test.png}

\footnotesize{Regressions of unadjusted and recentered market access growth on geographic features. Spatial-clustered standard errors in parentheses.}

\end{center}

\end{frame}

\begin{frame}[label=HsrRcScatter]{Recentered MA Doesn't Predict Employment Growth!}
	\includegraphics[width=\textwidth]{lecture_includes/emp_rc_nlink_binscatter.pdf} 
\end{frame}

\begin{frame}[label=HSRTable]{Adjusted Estimates of Market Access Effects}
	\begin{center}
	\includegraphics[width=0.9\textwidth]{lecture_includes/hsr_tab.png}
	
	\footnotesize{Regressions of log employment growth on log market access growth in 2007--2016. Spatial-clustered standard errors in parentheses; permutation-based 95\% CI in brackets}
	\end{center}
\end{frame}

\subsection{Medicaid Eligibility Effects}
\begin{frame}[label=ACA]{App. 2: Efficient Estimation of Medicaid Effects}
\vspace{-0.1cm}
\textbf{Setting}: U.S. Medicaid, partially expanded in 2014 under the ACA
\vspace{0.1cm}
	\begin{itemize}
	\item 19 of 43 states with low Medicaid coverage expanded to 138\% FPL

	\item View \bgSun{expansion decisions} as random across states with same-party governors, but not \bgElectricViolet{household demographics} or \bgElectricViolet{pre-2014 policy}

	\item Outcomes: Medicaid takeup and private insurance crowdout
\end{itemize}
\pause
We compare two estimators, both valid under the same assumptions:
\begin{itemize}
	\item Simulated IV: use state-level variation only (i.e. expansion dummy)
	\item Recentered IV: predict eligibility from expansion decisions \& non-random demographics, and recenter
\end{itemize}

\pause
Via non-random variation, recentered IV has $\approx 3$ times smaller SEs
\end{frame}

\begin{frame}{Estimates with Simulated vs. Recentered IV}
	\begin{center}
	\includegraphics[width=1\textwidth]{lecture_includes/aca_ss.png}
	\end{center}

\footnotesize{1\% ACS sample of non-disabled adults in 2013--14, diff-in-diff IV regressions using one of the two instruments. Controls include state and year fixed effects and an indicator for Republican governor interacted with year. State-clustered standard errors in parentheses; wild score bootstrap 95\% CI in brackets} 
\end{frame}

\section{Concluding Thoughts}
\begin{frame}{Conclusions}
In both linear SSIV and more elaborate settings, the most important thing is to decide \emph{ex ante} what identifying variation you want to use\smallskip\pause
\begin{itemize}
\item When leveraging a natural experiment, recentering (e.g. controlling for sum-of-shares, in linear SSIV) can help\smallskip\pause
\item Non-experimental assumptions (e.g. parallel trends) typically require other approaches\pause{} \smallskip
\item The source of variation can (should?) guide inference 
\end{itemize}\medskip\pause

After deciding + appropriately adjusting the analysis, try to falsify the identifying variation (\emph{ex post}) -- via balance or pre-trend tests\bigskip\pause

Much more work to be done on the various econometrics here!
\end{frame}


\begin{frame}{Keep Calm and SSIV On!}

\begin{center}
Good luck on your future adventures with SSIV!

\bigskip
\url{peter_hull@brown.edu}

\bigskip
\url{@instrumenthull}
\end{center}
\end{frame}

\end{document}
